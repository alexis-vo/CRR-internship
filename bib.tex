\documentclass[11pt]{article}
\usepackage[utf8]{inputenc}
\usepackage[T1]{fontenc}
\usepackage{hyperref}

\title{Modèle de Cox-Ross-Rubinstein\\\medskip Bibliographie annotée}
\author{Alexis VO\\Université Paris-Saclay}

\begin{document}
\maketitle

Cette bibliographie annotée présente les références que j'ai utilisé dans le cadre de mon stage au CMAP de l'Ecole polytechnique. Chaque référence est suivie d'une brève annotation qui résume son contenu et sa pertinence pour le sujet traité.

\vspace{1em}

\textbf{Renaud Bourles} \textit{Chapitre 9 - Le modèle de Cox-Ross-Rubinstein}. Mathematics for finance (2008-2010). 1st year Master Finance. Probabilités discrètes et modèles en temps discret.
Disponible en ligne : \href{http://renaud.bourles.perso.centrale-med.fr/MathsFi/Chap%209%20-%20Le%20modele%20Cox-Ross-Rubinstein.pdf}{lien vers le chapitre}.
\begin{itemize}
    \item Evaluation des options dans un cadre discret puis continu.
    \item Application à l'évaluation des options européennes et américaines, call/put.
    \item Découverte du modèle à une et deux périodes.
    \item Stratégie de couverture et absence d'arbitrage.
    \item Généralisation à $n$ périodes.
\end{itemize}
Bonne lecture introductive qui plonge instantanément dans le sujet.\\ 

\vspace{1em}

\end{document}