\documentclass[a4paper,10pt]{article}
\usepackage[utf8]{inputenc}
\usepackage[T1]{fontenc}
\usepackage[french]{babel}
\usepackage{multicol}
\usepackage[colorlinks=true, linkcolor=blue]{hyperref}
\usepackage{geometry}
\geometry{margin=2cm}

\usepackage{titling}
\renewcommand\maketitlehooka{\null\mbox{}\vfill}
\renewcommand\maketitlehookd{\vfill\null}

\title{\Huge{\textbf{Modèle de Cox-Ross-Rubinstein}}\\ \medskip
      \Huge{\textit{Glossaire}}\vspace*{0.7cm}}
\author{\LARGE{Alexis VO}\vspace{1cm}\\ \medskip
      Université Paris-Saclay\\École polytechnique}
\date{\vspace{0.2cm}\today}

% === BEGIN DOCUMENT ===
\begin{document}

\vspace{\fill}
  \maketitle
\vspace{\fill}

\newpage

\tableofcontents

\newpage

\begin{multicols}{2}

\section*{A}
\addcontentsline{toc}{section}{A}
\begin{itemize}
  \item \textbf{Arbitrage} : stratégie d’investissement qui permet de réaliser un profit sans risque, en exploitant les différences de prix entre deux ou plusieurs marchés.
  \item \textbf{Actif} : bien ou droit possédé par un individu ou une entreprise, qui peut générer des flux de trésorerie futurs.
  \item \textbf{Action} : titre de propriété représentant une part du capital d’une entreprise.
  \item \textbf{Ajustement} : modification apportée à un modèle ou à une stratégie pour mieux refléter la réalité du marché.
  \item \textbf{Analyse technique} : méthode d’évaluation des titres basée sur l’étude des graphiques et des tendances passées des prix.
\end{itemize}

\section*{B}
\addcontentsline{toc}{section}{B}
\begin{itemize}
  \item \textbf{Bourse} : marché organisé où s’échangent des titres financiers tels que les actions, les obligations et les produits dérivés.
  \item \textbf{Bénéfice} : gain réalisé par une entreprise après déduction de ses coûts et charges.
\end{itemize}

\section*{C}
\addcontentsline{toc}{section}{C}
\begin{itemize}
  \item \textbf{Capital} : somme d’argent investie dans une entreprise ou un actif, qui peut générer des revenus.
  \item \textbf{Cotation} : prix auquel un actif est échangé sur le marché.
  \item \textbf{Contrat à terme} : accord entre deux parties pour acheter ou vendre un actif à un prix fixé à une date future.
  \item \textbf{Call} : option d’achat qui donne le droit, mais pas l’obligation, d’acheter un actif à un prix déterminé avant une date d’échéance.
\end{itemize}

\section*{D}
\addcontentsline{toc}{section}{D}

\section*{E}
\addcontentsline{toc}{section}{E}

\section*{F}
\addcontentsline{toc}{section}{F}
\begin{itemize}
  \item \textbf{Finance} : domaine d’étude et de gestion des ressources monétaires dans le temps, en particulier dans des situations d’incertitude.
\end{itemize}

\section*{G}
\addcontentsline{toc}{section}{G}

\section*{H}
\addcontentsline{toc}{section}{H}

\section*{I}
\addcontentsline{toc}{section}{I}

\section*{J}
\addcontentsline{toc}{section}{J}

\section*{K}
\addcontentsline{toc}{section}{K}

\section*{L}
\addcontentsline{toc}{section}{L}

\section*{M}
\addcontentsline{toc}{section}{M}
\begin{itemize}
  \item \textbf{Model} : représentation simplifiée et théorique de l’évolution d’un phénomène financier.
\end{itemize}

\section*{N}
\addcontentsline{toc}{section}{N}

\section*{O}
\addcontentsline{toc}{section}{O}

\section*{P}
\addcontentsline{toc}{section}{P}
\begin{itemize}
  \item \textbf{Put} : option de vente qui donne le droit, mais pas l’obligation, de vendre un actif à un prix déterminé avant une date d’échéance.
\end{itemize}

\section*{Q}
\addcontentsline{toc}{section}{Q}

\section*{R}
\addcontentsline{toc}{section}{R}

\section*{S}
\addcontentsline{toc}{section}{S}

\section*{T}
\addcontentsline{toc}{section}{T}

\section*{U}
\addcontentsline{toc}{section}{U}

\section*{V}
\addcontentsline{toc}{section}{V}

\section*{W}
\addcontentsline{toc}{section}{W}

\section*{X}
\addcontentsline{toc}{section}{X}

\section*{Y}
\addcontentsline{toc}{section}{Y}

\section*{Z}
\addcontentsline{toc}{section}{Z}

\end{multicols}

\end{document}