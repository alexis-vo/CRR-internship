\documentclass[11pt]{article}
\usepackage[utf8]{inputenc}
\usepackage[T1]{fontenc}
\usepackage{amsmath, amssymb}
\usepackage{tcolorbox}
\usepackage{geometry}
\geometry{margin=2.5cm}

\usepackage{titling}
\renewcommand\maketitlehooka{\null\mbox{}\vfill}
\renewcommand\maketitlehookd{\vfill\null}

\title{\Huge{\textbf{Une introduction aux variables aléatoires}}\\ \medskip
      \Huge{\textit{Résumé}}\vspace*{0.7cm}}
\author{\LARGE{Alexis VO}\vspace{1cm}\\ \medskip
      Université Paris-Saclay\\École polytechnique}
\date{\vspace{0.2cm}\today}

\begin{document}

\maketitle
\newpage

\section{Qu'est-ce qu'une variable aléatoire ?}

Une \textbf{variable aléatoire} (abrégée VA) est une façon de modéliser une expérience aléatoire à l’aide d’une valeur numérique. Elle associe à chaque issue possible d’une expérience un nombre réel.

\begin{tcolorbox}[colback=green!5!white, colframe=green!50!black, title=Exemple (discret)]
Lancer un dé à 6 faces : on peut modéliser le résultat par une variable aléatoire \( X \) telle que \( X \in \{1,2,3,4,5,6\} \)
\end{tcolorbox}

Il existe deux grandes familles de variables aléatoires : \textbf{discrètes et continues}.

\section*{2. Variable aléatoire discrète}

Une VA discrète prend un nombre \textbf{fini ou dénombrable} de valeurs. Elle est définie par une \textbf{loi de probabilité} qui associe une probabilité à chaque valeur.

\[
\boxed{
\mathbb{P}(X = x_i)
\quad \text{avec} \quad \sum_i \mathbb{P}(X = x_i) = 1
}
\]

\section{Variable aléatoire continue}

Une VA continue peut prendre \textbf{toutes les valeurs} dans un intervalle de réels (par exemple \( [0, +\infty[ \), ou \( \mathbb{R} \)).

\begin{tcolorbox}[colback=red!5!white, colframe=red!50!black, title=Définition]
Une variable aléatoire \( X \) est \textbf{continue} s’il existe une fonction \( p(x) \), appelée \textit{densité de probabilité}, telle que pour tout intervalle \( [a,b] \) :
\[
\boxed{
\mathbb{P}(a \leq X \leq b) = \int_a^b p(x)\, dx
}
\]
\center et
\[
\boxed{
\int_{-\infty}^{+\infty} p(x)\, dx = 1
}
\]
\end{tcolorbox}

\subsection*{Remarques :}
\begin{itemize}
    \item Contrairement au cas discret, on n’a jamais \( \mathbb{P}(X = x) > 0 \). Pour toute valeur précise, \( \mathbb{P}(X = x) = 0 \).
    \item Ce qui est probable, ce n’est pas une valeur exacte, mais un \textbf{intervalle}.
\end{itemize}

\section*{4. Exemple classique : la loi normale}

La densité de la loi normale centrée réduite est :

\[
\boxed{
p(x) = \frac{1}{\sqrt{2\pi}} \exp\left(-\frac{x^2}{2}\right)
}
\]

\begin{itemize}
    \item Elle est symétrique par rapport à 0.
    \item Elle est très utilisée pour modéliser des phénomènes naturels, physiques ou économiques (par exemple les fluctuations boursières chez Bachelier).
\end{itemize}

\section*{5. Résumé}

\begin{tcolorbox}[colback=yellow!5!white, colframe=yellow!60!black, title=Résumé]
Une variable aléatoire continue :
\begin{itemize}
    \item prend ses valeurs dans un intervalle de réels ;
    \item est modélisée par une \textbf{densité de probabilité} \( p(x) \) ;
    \item permet de calculer les probabilités via une \textbf{intégrale} ;
    \item vérifie toujours \( \mathbb{P}(X = x) = 0 \) pour toute valeur fixe \( x \).
\end{itemize}
\end{tcolorbox}

\end{document}