\documentclass[a4paper,10pt]{article}
\usepackage[utf8]{inputenc}
\usepackage[T1]{fontenc}
\usepackage[colorlinks=true, linkcolor=black]{hyperref}
\usepackage{bookmark}
\usepackage[french]{babel}
\usepackage{geometry}
\usepackage{titlesec}
\titleformat{name=\section,numberless}[runin]{\normalsize\bfseries}{}{0pt}{}

\usepackage{titling}
\renewcommand\maketitlehooka{\null\mbox{}\vfill}
\renewcommand\maketitlehookd{\vfill\null}

\usepackage{tocloft}
\renewcommand{\contentsname}{}
\setlength{\cftbeforetoctitleskip}{0pt}
\setlength{\cftaftertoctitleskip}{0pt}

\title{\Huge{\textbf{Modèle de Cox-Ross-Rubinstein}}\\ \medskip
      \Huge{\textit{Bibliographie annotée}}\vspace*{0.7cm}}
\author{\LARGE{Alexis VO}\vspace{1cm}\\ \medskip
      Université Paris-Saclay\\École polytechnique}
\date{\vspace{0.2cm}\today}

% === BEGIN DOCUMENT ===
\begin{document}

\vspace{\fill}
  \maketitle
\vspace{\fill}

\newpage

Cette bibliographie annotée présente les références que j'ai utilisé dans le cadre de mon stage au CMAP de l'Ecole polytechnique. Chaque référence est suivie d'une brève annotation qui résume son contenu et sa pertinence pour le sujet traité.\\ \medskip

\begin{center}
    \tableofcontents    
\end{center}

\newpage

\vspace{1em}

\section*{Renaud Bourles\quad}
\phantomsection
\addcontentsline{toc}{section}{Renaud Bourles}

\textit{Chapitre 9 - Le modèle de Cox-Ross-Rubinstein}. Mathematics for finance (2008-2010). 1st year Master Finance. Probabilités discrètes et modèles en temps discret. \href{http://renaud.bourles.perso.centrale-med.fr/MathsFi/Chap%209%20-%20Le%20modele%20Cox-Ross-Rubinstein.pdf}{Consulter}.
\begin{itemize}
    \item Evaluation des options dans un cadre discret puis continu.
    \item Application à l'évaluation des options européennes et américaines, call/put.
    \item Découverte du modèle à une et deux périodes.
    \item Stratégie de couverture et absence d'arbitrage.
    \item Généralisation à $n$ périodes.
\end{itemize}
Bonne lecture introductive qui plonge instantanément dans le sujet.\\ 

\vspace{1em}

\section*{Isabelle Bajeux}
\phantomsection
\addcontentsline{toc}{section}{Isabelle Bajeux}
\textit{Gestion de portefeuille dans un modèle binomial}. Annale d'économie et de statistique, n°13, Jan-Mar 1989, p.49-76 (28 pages). \href{https://www.jstor.org/stable/20075729}{Consulter}.
\begin{itemize}
    \item Modélisation binomiale pour les actifs financiers.
    \item Construction et gestion de portefeuilles.
    \item Équilibre rendement-risque.
    \item Absence d’arbitrage et valorisation des options.
\end{itemize}
Lecture similaire à celle de Renaud Bourles, mais plus orientée vers la gestion de portefeuille.\\

\vspace{1em}

\section*{Binomial options pricing model}
\phantomsection
\addcontentsline{toc}{section}{Wikipedia}
\textit{Wikipédia}. The Free Encyclopedia. \href{https://en.wikipedia.org/wiki/Binomial_options_pricing_model?oldid=215677262}{Consulter}.
\begin{itemize}
    \item Présentation du modèle binomial dit de \textit{Cox-Ross-Rubinstein}.
    \item Explication des concepts clés : arbres binomiaux, évaluation des options, avec formules associées.
    \item Comparaison avec d'autres modèles d'évaluation des options, notamment Black-Scholes.
\end{itemize}
Lecture active pour une vue d'ensemble du modèle. Bien pour les concepts de base et les applications.\\

\vspace{1em}

\section*{Évaluation d'option}
\phantomsection
\addcontentsline{toc}{section}{Wikipedia}
\textit{Wikipédia}. The Free Encyclopedia. \href{https://fr.wikipedia.org/wiki/Évaluation_d%27option}{Consulter}.
\begin{itemize}
    \item Résumé clair et efficace de l'évaluation d'options.
    \item Définitions, valeur intrinsèque, valeur temps.
    \item Très clair, cet article est à lire pour mieux comprendre CRR.
\end{itemize}
Court article qui résume bien l'évaluation d'une option, simple et efficace pour bien comprendre CRR. Sa version anglaise est plus détaillée : \href{https://en.wikipedia.org/wiki/Valuation_of_options}{Consulter}

\section*{Options Trading: understanding option prices}
\phantomsection
\addcontentsline{toc}{section}{Youtube}
\textit{YouTube}. Vidéo de Sky View Trading. \href{https://youtu.be/MiybniIIvx0?si=ohfR6c83wdCrzKVJ}{Consulter}.
\begin{itemize}
    \item Explication des bases pour les options call/put.
    \item Notions de ITM / ATM / OTM
    \item Volatilité
\end{itemize}
Vidéo courte qui résume bien les points cités ci-dessus. À regarder pour se rappeler et non pas pour apprendre.

\section*{Investopedia}
\phantomsection
\addcontentsline{toc}{section}{Investopedia}
\textit{Dictionnaire des termes financiers} \href{https://www.investopedia.com/financial-term-dictionary-4769738}{Consulter}.


\end{document}