\documentclass[a4paper,11pt]{article}
\usepackage[utf8]{inputenc}
\usepackage[T1]{fontenc}
\usepackage[french]{babel}
\usepackage{geometry}
\usepackage{graphicx}



\usepackage{enumitem}
\usepackage{amsmath, amssymb}
\usepackage{titlesec}
\usepackage{xcolor}
\usepackage{listings}

\usepackage{hyperref}
\hypersetup{
    colorlinks=true,
    citecolor=black,
    filecolor=black,
    linkcolor=black,
    urlcolor=red
}

\usepackage{titling}
\renewcommand\maketitlehooka{\null\mbox{}\vfill}
\renewcommand\maketitlehookd{\vfill\null}

\title{\Huge{\textbf{Modèle de Cox-Ross-Rubinstein}}\\ \medskip
      \Huge{\textit{Journal de bord}}\vspace*{0.7cm}}
\author{\LARGE{Alexis VO}\vspace{1cm}\\ \medskip
      Université Paris-Saclay\\École polytechnique}
\date{\vspace{0.2cm}\today}

\begin{document}

\maketitle
\newpage
\tableofcontents
\newpage

%==========================================================
% JOUR 1
%==========================================================

\section{Jour 1, développement d’un outil de gestion de portefeuille avec options}
\noindent La matinée a été consacrée à :
\begin{itemize}
    \item la compréhension du modèle de Cox-Ross-Rubinstein (CRR).
    \item la mise en place d'une structure de projet.
\end{itemize}
L'après-midi midi a été consacrée à :
\begin{itemize}
    \item la création d'une application interactive avec Streamlit.
    \item l'implémentation de la valorisation des options européennes et américaines.
\end{itemize}
La préparation du stage a permis de poser les bases nécessaires pour débuter efficacement.

\subsection{Objectifs de la journée}
Créer une application interactive pour :
\begin{itemize}
    \item Valoriser des options européennes et américaines
    \item Simuler la réplication dynamique
    \item Intégrer une interface utilisateur avec Streamlit
\end{itemize}

\subsection{Architecture du projet}
Paradigme de développement : \textbf{Programmation Orientée Obet}.
\begin{verbatim}
portfolio-pricing-system/
|-- app.py
|-- main.py
|-- core/
|   |-- options.py
|   |-- option_factory.py
|-- models/
|   |-- binomial_model.py
|   |-- black_scholes.py
|-- portfolio/
|   |-- hedging.py
|   |-- portfolio.py
|   |-- replication.py
|-- tests/
|   |-- test_binomial_model.py
|-- utils/
|   |-- visualization.py
\end{verbatim}

\subsection{Exemples Modules implémentés}

\subsubsection{options.py}
Contient les classes pour les différentes options :
\begin{itemize}
    \item \texttt{Option}, classe de base
    \item \texttt{EuropeanCallOption}, \texttt{EuropeanPutOption}
    \item \texttt{AmericanCallOption}, \texttt{AmericanPutOption}
\end{itemize}

Chaque classe hérite d'\texttt{Option} et implémente \texttt{payoff()} ainsi que le flag \texttt{is\_american}.

\subsubsection{option\_factory.py}
Implémente un \texttt{OptionFactory} qui permet d’instancier dynamiquement des options à partir d’un nom chaîne de caractères, comme \texttt{european\_call}.

\subsubsection{binomial\_model.py}
Contient la fonction \texttt{binomial\_option\_pricing} :
\begin{itemize}
    \item Construction de l’arbre des prix
    \item Backward induction pour valoriser l’option
    \item Prise en compte de l’exercice anticipé pour les options américaines
\end{itemize}

\subsection{À suivre...}
\begin{itemize}
    \item Intégration de la visualisation du portefeuille (\texttt{matplotlib} ou \texttt{plotly})
    \item Affichage de l’arbre binomial ou de la stratégie de couverture
    \item Élargissement à un portefeuille multi-options
    \item Ajout d’un module d’export PDF ou Excel
    \item etc...
\end{itemize}

\subsection{Conclusion}
L’application est en place. L’architecture modulaire permettra une extension facile vers d'autres modèles de valorisation et vers une gestion de portefeuille plus complète.

\newpage

%==========================================================
% JOUR 2
%==========================================================

\section{Jour 2, CRR et Black-Scholes}
La journée a été consacrée à la lecture des premiers chapitres du polycopié \textit{Martingales pour la finance} et à la mise en place du TP1.1 - \textit{Le modèle de CRR}.

\subsection{Objectifs de la journée}
\begin{itemize}
    \item Lecture du polycopié \textit{Martingales pour la finance}.
    \item Faire le TP1.1 - \textit{Le modèle de CRR}.
    \item Continuer le développement de l’application interactive.
\end{itemize}
\subsection{Travail réalisé}
\begin{itemize}
    \item Compréhension des concepts mathématiques tels que \( u_n, d_n, h_n, q \), etc.
    \item Implémentation en Python des fonctions :
    \begin{itemize}
        \item \texttt{Sn} pour les prix possibles de l’actif à une date donnée.
        \item \texttt{Payoff} pour le profil de gain d’un call européen à maturité.
        \item \texttt{Calln} pour le prix d’une option via l’évaluation backward.
        \item \texttt{Deltan} pour le vecteur de couverture delta à chaque étape.
    \end{itemize}
    \item Étude de la convergence : tracé de l’écart relatif entre CRR et Black-Scholes pour différents \( n \).
    \item Rédaction d’un résumé en {\LaTeX} avec explications mathématiques, intuition des formules, et comparaison entre les deux modèles.
\end{itemize}

\subsection{À suivre...}
\begin{itemize}
    \item Étendre l’étude à d’autres types d’options (put européen, option asiatique...).
    \item Intégrer une étude numérique de sensibilité aux paramètres \( \sigma, r, T \) (analyses dites \og greeks \fg).
    \item Ajouter des cas pratiques et graphiques interactifs pour illustrer le comportement de la couverture dynamique.
\end{itemize}

\subsection{Conclusion}
Durant cette deuxième journée, j'ai pu implémenter l’intégralité du modèle binomial de Cox-Ross-Rubinstein. J'ai également pu vérifier sa cohérence avec la formule continue de Black-Scholes. La compréhension de la couverture dynamique - \textit{delta hedging} - a été approfondie, et les outils numériques sont désormais en place pour explorer des cas plus complexes. Le lien entre les modèles discrets et continus a été mis en évidence à travers l’étude de la convergence.

\newpage

%==========================================================
% JOUR N
%==========================================================

\section{Jour N, ...}
\subsection{Objectifs de la journée}
\begin{itemize}
    \item ...
    \item ...
    \item ...
\end{itemize}
\subsection{...}
\subsection{À suivre...}
\subsection{Conclusion}
\newpage
\end{document}