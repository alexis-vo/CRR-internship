\documentclass[a4paper,11pt]{article}
\usepackage[utf8]{inputenc}
\usepackage[T1]{fontenc}
\usepackage[french]{babel}
\usepackage{geometry}
\usepackage{graphicx}



\usepackage{enumitem}
\usepackage{amsmath, amssymb}
\usepackage{titlesec}
\usepackage{xcolor}
\usepackage{listings}

\usepackage{hyperref}
\hypersetup{
    colorlinks=true,
    citecolor=black,
    filecolor=black,
    linkcolor=black,
    urlcolor=red
}

\usepackage{titling}
\renewcommand\maketitlehooka{\null\mbox{}\vfill}
\renewcommand\maketitlehookd{\vfill\null}

\title{\Huge{\textbf{Modèle de Cox-Ross-Rubinstein}}\\ \medskip
      \Huge{\textit{Journal de bord}}\vspace*{0.7cm}}
\author{\LARGE{Alexis VO}\vspace{1cm}\\ \medskip
      Université Paris-Saclay\\École polytechnique}
\date{\vspace{0.2cm}\today}

\begin{document}

\maketitle
\newpage
\tableofcontents
\newpage

%==========================================================
% JOUR 1
%==========================================================

\section{Jour 1, développement d’un outil de gestion de portefeuille avec options}
\noindent La matinée a été consacrée à :
\begin{itemize}
    \item la compréhension du modèle de Cox-Ross-Rubinstein (CRR).
    \item la mise en place d'une structure de projet.
\end{itemize}
L'après-midi midi a été consacrée à :
\begin{itemize}
    \item la création d'une application interactive avec Streamlit.
    \item l'implémentation de la valorisation des options européennes et américaines.
\end{itemize}
La préparation du stage a permis de poser les bases nécessaires pour débuter efficacement.

\subsection{Objectifs de la journée}
Créer une application interactive pour :
\begin{itemize}
    \item Valoriser des options européennes et américaines
    \item Simuler la réplication dynamique
    \item Intégrer une interface utilisateur avec Streamlit
\end{itemize}

\subsection{Architecture du projet}
Paradigme de développement : \textbf{Programmation Orientée Objet}.
\begin{verbatim}
portfolio-pricing-system/
|-- app.py
|-- main.py
|-- core/
|   |-- options.py
|   |-- option_factory.py
|-- models/
|   |-- binomial_model.py
|   |-- black_scholes.py
|-- portfolio/
|   |-- hedging.py
|   |-- portfolio.py
|   |-- replication.py
|-- tests/
|   |-- test_binomial_model.py
|-- utils/
|   |-- visualization.py
\end{verbatim}

\subsection{Exemples Modules implémentés}

\subsubsection{options.py}
Contient les classes pour les différentes options :
\begin{itemize}
    \item \texttt{Option}, classe de base
    \item \texttt{EuropeanCallOption}, \texttt{EuropeanPutOption}
    \item \texttt{AmericanCallOption}, \texttt{AmericanPutOption}
\end{itemize}

Chaque classe hérite d'\texttt{Option} et implémente \texttt{payoff()} ainsi que le flag \texttt{is\_american}.

\subsubsection{option\_factory.py}
Implémente un \texttt{OptionFactory} qui permet d’instancier dynamiquement des options à partir d’un nom chaîne de caractères, comme \texttt{european\_call}.

\subsubsection{binomial\_model.py}
Contient la fonction \texttt{binomial\_option\_pricing} :
\begin{itemize}
    \item Construction de l’arbre des prix
    \item Backward induction pour valoriser l’option
    \item Prise en compte de l’exercice anticipé pour les options américaines
\end{itemize}

\subsection{À suivre...}
\begin{itemize}
    \item Intégration de la visualisation du portefeuille (\texttt{matplotlib} ou \texttt{plotly})
    \item Affichage de l’arbre binomial ou de la stratégie de couverture
    \item Élargissement à un portefeuille multi-options
    \item Ajout d’un module d’export PDF ou Excel
    \item etc...
\end{itemize}

\subsection{Conclusion}
L’application est en place. L’architecture modulaire permettra une extension facile vers d'autres modèles de valorisation et vers une gestion de portefeuille plus complète.

\newpage

%==========================================================
% JOUR 2
%==========================================================

\section{Jour 2, CRR et Black-Scholes}
La journée a été consacrée à la lecture des premiers chapitres du polycopié \textit{Martingales pour la finance} et à la mise en place du TP1.1 - \textit{Le modèle de CRR}.

\subsection{Objectifs de la journée}
\begin{itemize}
    \item Lecture du polycopié \textit{Martingales pour la finance}.
    \item Faire le TP1.1 - \textit{Le modèle de CRR}.
    \item Continuer le développement de l’application interactive.
\end{itemize}
\subsection{Travail réalisé}
\begin{itemize}
    \item Compréhension des concepts mathématiques tels que \( u_n, d_n, h_n, q \), etc.
    \item Implémentation en Python des fonctions :
    \begin{itemize}
        \item \texttt{Sn} pour les prix possibles de l’actif à une date donnée.
        \item \texttt{Payoff} pour le profil de gain d’un call européen à maturité.
        \item \texttt{Calln} pour le prix d’une option via l’évaluation backward.
        \item \texttt{Deltan} pour le vecteur de couverture delta à chaque étape.
    \end{itemize}
    \item Étude de la convergence : tracé de l’écart relatif entre CRR et Black-Scholes pour différents \( n \).
    \item Rédaction d’un résumé en {\LaTeX} avec explications mathématiques, intuition des formules, et comparaison entre les deux modèles.
\end{itemize}

\subsection{À suivre...}
\begin{itemize}
    \item Étendre l’étude à d’autres types d’options (put européen, option asiatique...).
    \item Intégrer une étude numérique de sensibilité aux paramètres \( \sigma, r, T \) (analyses dites \og greeks \fg).
    \item Ajouter des cas pratiques et graphiques interactifs pour illustrer le comportement de la couverture dynamique.
\end{itemize}

\subsection{Conclusion}
Durant cette deuxième journée, j'ai pu implémenter l’intégralité du modèle binomial de Cox-Ross-Rubinstein. J'ai également pu vérifier sa cohérence avec la formule continue de Black-Scholes. La compréhension de la couverture dynamique - \textit{delta hedging} - a été approfondie, et les outils numériques sont désormais en place pour explorer des cas plus complexes. Le lien entre les modèles discrets et continus a été mis en évidence à travers l’étude de la convergence.

\newpage

%==========================================================
% JOUR 3
%==========================================================

\section{Jour 3, rédaction d'un cours sur le modèle CRR}
\subsection{Objectifs de la journée}
\begin{itemize}
    \item Développer une stratégie de couverture dynamique.
    \item Poursiuvre l'intégration de la visualisation des résultats.
    \item Rédaction d'un cours sur le modèle de Cox-Ross-Rubinstein.
\end{itemize}
\subsection{Contenu abordé}

\subsubsection{Arbres binomiaux}
\begin{itemize}
    \item Arbre à un pas : valeurs numériques illustrant un actif pouvant monter ou baisser, avec calcul du prix d’un call.
    \item Arbre symbolique généré avec TikZ : noeuds représentant $S_0$, $S_u$, $S_d$, puis généralisation à $t = n$.
    \item Arbre centré sur les valeurs de l’option : $V_0$, $V_u$, $V_d$, flèches annotées par $q$ et $1-q$.
\end{itemize}

\subsubsection{Théorème d’évaluation neutre au risque}
\begin{itemize}
    \item Explication de la formule $V_0 = \frac{1}{1+r}(q V_u + (1 - q)V_d)$.
    \item Définition et intuition derrière la probabilité neutre au risque : $q = \frac{1 + r - d}{u - d}$.
    \item Illustration graphique via TikZ centrée sur la valeur de l’option.
\end{itemize}

\subsubsection{Portefeuille répliquant}
Début d'intuition graphique recherchée : représenter comment une combinaison d'actif sans risque (bond) et d’actif risqué permet de reproduire le payoff d’une option.

\subsubsection{Taux sans risque}
Clarification : le taux $r$ représente le rendement d’un actif sans risque, typiquement un \textit{bond}.

\subsubsection{Rédaction d’un commentaire d’interprétation d’un prix de call}

\subsubsection{Position courte}
\begin{itemize}
    \item Définition simple : vendre un actif qu’on ne détient pas, en espérant le racheter plus bas.
    \item Risques, fonctionnement, et illustration avec un exemple clair.
\end{itemize}

\subsection{À suivre...}
\begin{itemize}
    \item Construire un portefeuille répliquant dans un modèle binomial à un pas.
    \item Étendre les arbres binomiaux à plusieurs périodes ($n$ étapes).
    \item Implémenter les calculs de prix d’options selon le modèle CRR dans l'application interactives déjà prêtes (Streamlit).
    \item Étudier les notions de delta hedging et le lien avec la réplication.
    \item Approfondir le lien entre absence d’arbitrage et probabilités neutres au risque.
\end{itemize}

\subsection{Conclusion}
Cette troisième journée a permis de consolider les pour comprendre le modèle de CRR et les instruments dérivés simples comme le call européen (que j'ai donc mieux compris). Les exemples chiffrés et arbres binomiaux ont facilité l’appropriation de ces concepts. Les prochaines étapes visent à étendre ce cadre à plusieurs périodes et à mettre en œuvre les outils numériques nécessaires à une simulation complète (avec Streamlit par exemple).

\newpage

%==========================================================
% JOUR 4
%==========================================================

\section{Jour 4, fin du cours sur le modèle CRR}

\subsection{Objectifs de la journée}
\begin{itemize}
    \item Comprendre comment construire un portefeuille répliquant pour évaluer un produit dérivé.
    \item Introduire la stratégie de couverture (hedging) via la réplication.
    \item Définir la probabilité neutre au risque et comprendre son rôle.
    \item Relier la notion de \(\Delta\)-hedging à la dérivée du prix de l’option.
\end{itemize}

\subsection{Contenu abordé}

\subsubsection*{Modèle binomial à une période}
\begin{itemize}
    \item L’actif risqué vaut \(S_0\) aujourd’hui.
    \item Il peut évoluer vers \(S_0 u\) (hausse) ou \(S_0 d\) (baisse) à la prochaine période.
    \item Il existe un actif sans risque avec un taux \(r\).
\end{itemize}

\subsubsection*{Portefeuille répliquant}
\begin{itemize}
    \item On construit un portefeuille de \(\Delta\) actions et \(B\) obligations.
    \item Objectif : faire en sorte que ce portefeuille reproduise les gains du produit dérivé dans les deux cas futurs.
    \item Valeur initiale : \(V_0 = \Delta S_0 + B\).
    \item Valeur finale en cas de hausse : \(V_u = \Delta S_0 u + B(1 + r)\).
    \item Valeur finale en cas de baisse : \(V_d = \Delta S_0 d + B(1 + r)\).
\end{itemize}

\subsubsection*{Résolution du système}
On résout les deux équations :
\[
\Delta = \frac{V_u - V_d}{S_0 (u - d)}, \quad B = \frac{u V_d - d V_u}{(1 + r)(u - d)}
\]

\subsubsection*{Lien avec la couverture}
\begin{itemize}
    \item Cette stratégie est appelée \textbf{stratégie de couverture}.
    \item On annule le risque de marché en prenant une position opposée à celle de l’option.
    \item \(\Delta\) est appelé \textbf{delta} car il représente la dérivée du prix de l’option par rapport au sous-jacent.
    \item Cela fonde la notion de \textbf{delta-hedging}.
\end{itemize}

\subsubsection*{Probabilité neutre au risque}
\[
q = \frac{(1 + r) - d}{u - d}, \quad 1 - q = \frac{u - (1 + r)}{u - d}
\]

\subsubsection*{Formule de valorisation}
On obtient finalement :
\[
\boxed{V_0 = \frac{1}{1 + r} \left( q \cdot V_u + (1 - q) \cdot V_d \right)}
\]
Ce qui correspond à la valeur actualisée de l'espérance du produit dérivé sous la probabilité neutre au risque.

\subsection{À suivre...}
\begin{itemize}
    \item Développer la convergence du modèle binomial vers le modèle de Black-Scholes.
\end{itemize}

\subsection{Conclusion}
\begin{itemize}
    \item Le modèle CRR permet d’évaluer les options sans arbitrage par réplication.
    \item Le portefeuille répliquant constitue une stratégie de couverture efficace.
    \item La notion de \(\Delta\)-hedging est centrale en gestion du risque.
    \item La probabilité neutre au risque offre un cadre probabiliste puissant pour la valorisation.
\end{itemize}

\newpage

%==========================================================
% JOUR 5
%==========================================================

\section{Jour 5, Implémentation du portefeuille}
\subsection{Objectifs de la journée}
\begin{itemize}
    \item Implémenter un système de gestion de portefeuille (portefeuille multi-options)
    \item Intégrer une stratégie de couverture (delta hedging) au sein du portefeuille
    \item Ajouter des tests unitaires pour valider le bon fonctionnement des composants
\end{itemize}

\subsection{Travail réalisé}
\begin{itemize}
    \item Création du fichier \texttt{portfolio.py} pour gérer un ensemble de positions sur options.
    \item Implémentation du fichier \texttt{hedging.py} pour calculer les deltas et recommander une couverture.
    \item Résolution d’un bug lié à l’attribut \texttt{pricing\_model} manquant dans le portefeuille.
    \item Écriture de tests unitaires
\end{itemize}

\subsection{À suivre...}
\begin{itemize}
    \item Ajouter des scénarios dynamiques simulant l’évolution du marché (trajectoire du spot)
    \item Implémenter une stratégie de rebalancement du hedge à chaque pas de temps
    \item Intégrer les frais de transaction dans la stratégie de couverture
    \item Optimiser la visualisation avec des animations ou des courbes interactives
\end{itemize}

\subsection{Conclusion}
La journée a permis d’implémenter les bases d’un système cohérent de valorisation de portefeuille avec couverture delta.

\newpage

%==========================================================
% JOUR 6
%==========================================================

\section{Jour 6, Préparation entretien}
\subsection{Objectifs de la journée}
\begin{itemize}
    \item Notion de probabilité risque-neutre.
    \item Question de la volatilité constante.
    \item Convergence du modèle CRR vers Black-Scholes.
    \item Préparer l'entretien avec le tuteur.
\end{itemize}

\subsection{Notions abordées}

\begin{itemize}
    \item \textbf{Marché frictionless} : absence de coûts de transaction, liquidité parfaite, absence de contraintes sur les ventes à découvert.
    
    \item \textbf{Probabilité risque-neutre} :
    \[
        p = \frac{e^{r \Delta t} - d}{u - d}
    \]
    utilisée pour valoriser les actifs en actualisant les espérances sous une mesure fictive où le taux de croissance est le taux sans risque \(r\).

    \item \textbf{Volatilité constante} : dans le modèle CRR, les facteurs \(u = e^{\sigma \sqrt{\Delta t}}\) et \(d = e^{-\sigma \sqrt{\Delta t}}\) impliquent une volatilité constante au cours du temps.

    \item \textbf{Convergence} : le modèle binomial converge en loi vers un mouvement brownien géométrique :
    \[
        \ln S_T \xrightarrow{\text{loi}} \mathcal{N} \left( \ln S_0 + \left(r - \frac{\sigma^2}{2}\right)T,\; \sigma^2 T \right)
    \]

    \item \textbf{Formules d’actualisation} :
    \begin{align*}
        C_0 &= e^{-r \Delta t} \cdot \left( p C_u + (1 - p) C_d \right) \quad \text{(formule continue)} \\
        V_0 &= \frac{1}{1 + r} \cdot \left( q V_u + (1 - q) V_d \right) \quad \text{(formule discrète)}
    \end{align*}

    \item \textbf{Approches du modèle binomial} :
    \begin{itemize}
        \item \textit{Forward tree} : construction de l’arbre des prix.
        \item \textit{Backward induction} : valorisation des options par récurrence à rebours à partir des valeurs terminales.
    \end{itemize}
\end{itemize}

\subsection{Questions de recherche à discuter}

\begin{itemize}
    \item Peut-on adapter le modèle CRR à une volatilité non constante ?
    \item Quelle est la vitesse de convergence vers le modèle de Black-Scholes ?
    \item Que deviennent les stratégies de couverture delta sous contraintes réelles ?
    \item Comment le modèle s’étend-il à plusieurs actifs ou à des options exotiques ?
\end{itemize}

\subsection{À suivre...}

\begin{itemize}
    \item Étudier en détail la couverture dynamique dans CRR.
    \item Implémenter le modèle en Python pour simuler la convergence.
    \item Lire sur les extensions multipériodes ou multinodales du modèle binomial et surtout la convergence vers Black-Scholes.
\end{itemize}

\subsection{Conclusion}

La journée a permis de bien clarifier les fondements du modèle CRR, d'approfondir les bases théoriques de la valorisation des options par arbitrage, et d'identifier plusieurs directions de recherche intéressantes. La compréhension de la probabilité risque-neutre et de la convergence vers Black-Scholes est maintenant plus claire. Des travaux de modélisation numérique sont prévus pour approfondir.
\newpage

%==========================================================
% JOUR 7
%==========================================================

\section{Jour 7, Entretien et séminaire (digression)}
\subsection{Objectifs de la journée}
\begin{itemize}
    \item Entretien avec le tuteur
    \item Séminaire \textit{Generative Modeling for Finance}.
    \item Résumer des exposés du séminaire.
\end{itemize}

\subsection{Contenu de la journée}

\begin{itemize}
    \item \textbf{Exposé 1 : Flow matching}
    \begin{itemize}
        \item Problème : transformer une loi source \( p \) en une loi cible \( q \) en apprenant une transformation différentiable \( \varphi_\theta \).
        \item Méthode : apprentissage de la densité \( p^\theta \) à partir d’une équation de continuité du type Liouville.
        \item Implémentation : modèle de champ de vitesse \( u_t^\theta \) paramétré par un réseau de neurones et intégré numériquement.
    \end{itemize}
    
    \item \textbf{Exposé 2 : Conditional Flow Matching}
    \begin{itemize}
        \item Objectif : apprendre un flot conditionné sur l’état final \( x_1 \), pour générer efficacement les dynamiques de transition.
        \item Résultat central : les pertes de flow matching marginal et conditionnel ont le même gradient.
        \item Perspectives d’application : modélisation de dynamiques de marché conditionnelles en finance quantitative.
    \end{itemize}

    \item \textbf{Travail personnel} : rédaction d’un résumé des deux exposés et compréhension pro-active post-séminaire.
\end{itemize}

\subsection{À suivre...}
La poursuite de cette étude n'est pas au programme. Nous reviendrons sur les objectifs du stage ayant pour sujet principal le modèle binomial de Cox-Ross-Rubinstein.

\subsection{Conclusion}
Journée dense et riche en contenu théorique. Les exposés ont permis de découvrir une approche moderne du transport de mesures, basée sur l’apprentissage de champs de vitesse via des équations différentielles. La rédaction du résumé a facilité l’appropriation des concepts.

\newpage

%==========================================================
% JOUR 8
%==========================================================

\section{Jour 8, Implémentations}

\subsection{Objectifs de la journée}
\begin{itemize}
    \item Déboguer les erreurs du modèle binomial pour le pricing d’options européennes et américaines.
    \item Corriger et faire passer les tests unitaires dans \texttt{test\_binomial.py} et \texttt{test\_portfolio.py}.
    \item S’assurer que les options soient correctement valorisées dans un portefeuille en utilisant une structure flexible.
\end{itemize}

\subsection{Débogage du modèle binomial}

J'ai rencontré une série d'erreurs \texttt{TypeError} dues à un champ \texttt{spot} non défini dans les objets d’options passés à la fonction \texttt{binomial\_option\_pricing}. Ces erreurs ont été corrigées en modifiant la fonction pour extraire systématiquement les paramètres depuis l’objet option :

\begin{itemize}
    \item \texttt{spot = option.spot}
    \item \texttt{strike = option.strike}
\end{itemize}

Cela a permis de rendre la fonction indépendante du passage d’arguments externes incohérents ou manquants.

\subsection{Correction des tests du portefeuille}

Dans le fichier \texttt{test\_portfolio.py}, deux erreurs bloquaient les tests :
\begin{itemize}
    \item \texttt{TypeError} lié à un argument \texttt{spot\_dict} manquant.
    \item \texttt{KeyError: 'UNKNOWN'} provoqué par un attribut \texttt{asset\_name} non renseigné dans les options.
\end{itemize}

J'ai corrigé ces erreurs en spécifiant explicitement \texttt{asset\_name = "AAPL"} ou \texttt{"TSLA"} dans les objets \texttt{EuropeanOption} et \texttt{AmericanOption}. Passant un \texttt{spot\_dict} cohérent à toutes les fonctions dépendantes du portefeuille. Tous les tests passent désormais sans erreur.

\subsection{À suivre...}
\begin{itemize}
    \item Étendre les tests à d’autres types de portefeuilles, y compris des paniers d’options hétérogènes.
    \item Implémenter une gestion plus robuste des erreurs dans le pricing model (ex. : option non prenable par le modèle).
    \item Commencer la structuration de l’interface utilisateur (ex: Streamlit).
\end{itemize}

\subsection{Conclusion}

Aujourd'hui j'ai pu fiabiliser le modèle de valorisation binomiale et assurer son intégration correcte dans le système de portefeuille. Le pipeline de test unitaire est désormais stable pour les cas de base, ce qui constitue un socle solide pour ajouter de nouvelles fonctionnalités dans les jours à venir.

\newpage

%==========================================================
% JOUR 9
%==========================================================

\section{Jour 9, Implémentations}

\subsection{Objectifs de la journée}
\begin{itemize}
    \item Générer un arbre binomial de valorisation d'option dans le terminal puis avec Graphviz.
    \item Mettre en évidence les points d'exercice anticipé dans le cas d'une option américaine.
    \item Ajouter une légende visuelle pour clarifier les couleurs utilisées dans le graphe.
\end{itemize}

\subsection{Travail réalisé}
\begin{itemize}
    \item Ajout d’un affichage texte de l’arbre binomial : valeurs de l’actif et de l’option à chaque nœud.
    \item Intégration de Graphviz pour produire une représentation graphique de l’arbre.
    \item Mise en couleur des nœuds selon que l’option est exercée ou non (exercice anticipé en rouge, conservation en bleu).
    \item Tests pour identifier des cas où l’exercice anticipé peut survenir mais pas systématiquement.
\end{itemize}

\subsection{À suivre...}
\begin{itemize}
    \item Permettre l'affichage interactif ou animé de l’arbre (via une interface web avec Streamlit).
    \item Étendre la visualisation à un portefeuille d’options (superposition ou juxtaposition d’arbres).
    \item Tester avec différents types d’options : put américain, call américain, européens...
\end{itemize}

\subsection{Conclusion}
Valorisation par arbre binomial en visualisant le cheminement de la valeur de l’option. Les ajustements esthétiques et logiques apportés aux graphes ouvrent la voie à de futures extensions interactives/pédagogiques.
\newpage

%==========================================================
% JOUR 10
%==========================================================

\section{Jour 10, Implémentations}
\subsection{Objectifs de la journée}
\begin{itemize}
    \item ...
\end{itemize}
\subsection{...}
\subsection{À suivre...}
\subsection{Conclusion}
\newpage

%==========================================================
% JOUR N
%==========================================================

\section{Jour N, ...}
\subsection{Objectifs de la journée}
\begin{itemize}
    \item ...
\end{itemize}
\subsection{...}
\subsection{À suivre...}
\subsection{Conclusion}
\newpage


%==========================================================
\end{document}