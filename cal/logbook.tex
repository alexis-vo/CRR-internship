\documentclass[a4paper,11pt]{article}
\usepackage[utf8]{inputenc}
\usepackage[T1]{fontenc}
\usepackage[french]{babel}
\usepackage{geometry}
\usepackage{graphicx}



\usepackage{enumitem}
\usepackage{amsmath, amssymb}
\usepackage{titlesec}
\usepackage{xcolor}
\usepackage{listings}

\usepackage{hyperref}
\hypersetup{
    colorlinks=true,
    citecolor=black,
    filecolor=black,
    linkcolor=black,
    urlcolor=red
}

\usepackage{titling}
\renewcommand\maketitlehooka{\null\mbox{}\vfill}
\renewcommand\maketitlehookd{\vfill\null}

\title{\Huge{\textbf{Modèle de Cox-Ross-Rubinstein}}\\ \medskip
      \Huge{\textit{Journal de bord}}\vspace*{0.7cm}}
\author{\LARGE{Alexis VO}\vspace{1cm}\\ \medskip
      Université Paris-Saclay\\École polytechnique}
\date{\vspace{0.2cm}\today}

\begin{document}

\maketitle
\newpage
\tableofcontents
\newpage

%==========================================================
% JOUR 1
%==========================================================

\section{Jour 1, développement d’un outil de gestion de portefeuille avec options}
\noindent La matinée a été consacrée à :
\begin{itemize}
    \item la compréhension du modèle de Cox-Ross-Rubinstein (CRR).
    \item la mise en place d'une structure de projet.
\end{itemize}
L'après-midi midi a été consacrée à :
\begin{itemize}
    \item la création d'une application interactive avec Streamlit.
    \item l'implémentation de la valorisation des options européennes et américaines.
\end{itemize}
La préparation du stage a permis de poser les bases nécessaires pour débuter efficacement.

\subsection{Objectifs de la journée}
Créer une application interactive pour :
\begin{itemize}
    \item Valoriser des options européennes et américaines
    \item Simuler la réplication dynamique
    \item Intégrer une interface utilisateur avec Streamlit
\end{itemize}

\subsection{Architecture du projet}
Paradigme de développement : \textbf{Programmation Orientée Objet}.
\begin{verbatim}
portfolio-pricing-system/
|-- app.py
|-- main.py
|-- core/
|   |-- options.py
|   |-- option_factory.py
|-- models/
|   |-- binomial_model.py
|   |-- black_scholes.py
|-- portfolio/
|   |-- hedging.py
|   |-- portfolio.py
|   |-- replication.py
|-- tests/
|   |-- test_binomial_model.py
|-- utils/
|   |-- visualization.py
\end{verbatim}

\subsection{Exemples Modules implémentés}

\subsubsection{options.py}
Contient les classes pour les différentes options :
\begin{itemize}
    \item \texttt{Option}, classe de base
    \item \texttt{EuropeanCallOption}, \texttt{EuropeanPutOption}
    \item \texttt{AmericanCallOption}, \texttt{AmericanPutOption}
\end{itemize}

Chaque classe hérite d'\texttt{Option} et implémente \texttt{payoff()} ainsi que le flag \texttt{is\_american}.

\subsubsection{option\_factory.py}
Implémente un \texttt{OptionFactory} qui permet d’instancier dynamiquement des options à partir d’un nom chaîne de caractères, comme \texttt{european\_call}.

\subsubsection{binomial\_model.py}
Contient la fonction \texttt{binomial\_option\_pricing} :
\begin{itemize}
    \item Construction de l’arbre des prix
    \item Backward induction pour valoriser l’option
    \item Prise en compte de l’exercice anticipé pour les options américaines
\end{itemize}

\subsection{À suivre...}
\begin{itemize}
    \item Intégration de la visualisation du portefeuille (\texttt{matplotlib} ou \texttt{plotly})
    \item Affichage de l’arbre binomial ou de la stratégie de couverture
    \item Élargissement à un portefeuille multi-options
    \item Ajout d’un module d’export PDF ou Excel
    \item etc...
\end{itemize}

\subsection{Conclusion}
L’application est en place. L’architecture modulaire permettra une extension facile vers d'autres modèles de valorisation et vers une gestion de portefeuille plus complète.

\newpage

%==========================================================
% JOUR 2
%==========================================================

\section{Jour 2, CRR et Black-Scholes}
La journée a été consacrée à la lecture des premiers chapitres du polycopié \textit{Martingales pour la finance} et à la mise en place du TP1.1 - \textit{Le modèle de CRR}.

\subsection{Objectifs de la journée}
\begin{itemize}
    \item Lecture du polycopié \textit{Martingales pour la finance}.
    \item Faire le TP1.1 - \textit{Le modèle de CRR}.
    \item Continuer le développement de l’application interactive.
\end{itemize}
\subsection{Travail réalisé}
\begin{itemize}
    \item Compréhension des concepts mathématiques tels que \( u_n, d_n, h_n, q \), etc.
    \item Implémentation en Python des fonctions :
    \begin{itemize}
        \item \texttt{Sn} pour les prix possibles de l’actif à une date donnée.
        \item \texttt{Payoff} pour le profil de gain d’un call européen à maturité.
        \item \texttt{Calln} pour le prix d’une option via l’évaluation backward.
        \item \texttt{Deltan} pour le vecteur de couverture delta à chaque étape.
    \end{itemize}
    \item Étude de la convergence : tracé de l’écart relatif entre CRR et Black-Scholes pour différents \( n \).
    \item Rédaction d’un résumé en {\LaTeX} avec explications mathématiques, intuition des formules, et comparaison entre les deux modèles.
\end{itemize}

\subsection{À suivre...}
\begin{itemize}
    \item Étendre l’étude à d’autres types d’options (put européen, option asiatique...).
    \item Intégrer une étude numérique de sensibilité aux paramètres \( \sigma, r, T \) (analyses dites \og greeks \fg).
    \item Ajouter des cas pratiques et graphiques interactifs pour illustrer le comportement de la couverture dynamique.
\end{itemize}

\subsection{Conclusion}
Durant cette deuxième journée, j'ai pu implémenter l’intégralité du modèle binomial de Cox-Ross-Rubinstein. J'ai également pu vérifier sa cohérence avec la formule continue de Black-Scholes. La compréhension de la couverture dynamique - \textit{delta hedging} - a été approfondie, et les outils numériques sont désormais en place pour explorer des cas plus complexes. Le lien entre les modèles discrets et continus a été mis en évidence à travers l’étude de la convergence.

\newpage

%==========================================================
% JOUR 3
%==========================================================

\section{Jour 3, rédaction d'un cours sur le modèle CRR}
\subsection{Objectifs de la journée}
\begin{itemize}
    \item Développer une stratégie de couverture dynamique.
    \item Poursiuvre l'intégration de la visualisation des résultats.
    \item Rédaction d'un cours sur le modèle de Cox-Ross-Rubinstein.
\end{itemize}
\subsection{Contenu abordé}

\subsubsection{Arbres binomiaux}
\begin{itemize}
    \item Arbre à un pas : valeurs numériques illustrant un actif pouvant monter ou baisser, avec calcul du prix d’un call.
    \item Arbre symbolique généré avec TikZ : noeuds représentant $S_0$, $S_u$, $S_d$, puis généralisation à $t = n$.
    \item Arbre centré sur les valeurs de l’option : $V_0$, $V_u$, $V_d$, flèches annotées par $q$ et $1-q$.
\end{itemize}

\subsubsection{Théorème d’évaluation neutre au risque}
\begin{itemize}
    \item Explication de la formule $V_0 = \frac{1}{1+r}(q V_u + (1 - q)V_d)$.
    \item Définition et intuition derrière la probabilité neutre au risque : $q = \frac{1 + r - d}{u - d}$.
    \item Illustration graphique via TikZ centrée sur la valeur de l’option.
\end{itemize}

\subsubsection{Portefeuille répliquant}
Début d'intuition graphique recherchée : représenter comment une combinaison d'actif sans risque (bond) et d’actif risqué permet de reproduire le payoff d’une option.

\subsubsection{Taux sans risque}
Clarification : le taux $r$ représente le rendement d’un actif sans risque, typiquement un \textit{bond}.

\subsubsection{Rédaction d’un commentaire d’interprétation d’un prix de call}

\subsubsection{Position courte}
\begin{itemize}
    \item Définition simple : vendre un actif qu’on ne détient pas, en espérant le racheter plus bas.
    \item Risques, fonctionnement, et illustration avec un exemple clair.
\end{itemize}

\subsection{À suivre...}
\begin{itemize}
    \item Construire un portefeuille répliquant dans un modèle binomial à un pas.
    \item Étendre les arbres binomiaux à plusieurs périodes ($n$ étapes).
    \item Implémenter les calculs de prix d’options selon le modèle CRR dans l'application interactives déjà prêtes (Streamlit).
    \item Étudier les notions de delta hedging et le lien avec la réplication.
    \item Approfondir le lien entre absence d’arbitrage et probabilités neutres au risque.
\end{itemize}

\subsection{Conclusion}
Cette troisième journée a permis de consolider les pour comprendre le modèle de CRR et les instruments dérivés simples comme le call européen (que j'ai donc mieux compris). Les exemples chiffrés et arbres binomiaux ont facilité l’appropriation de ces concepts. Les prochaines étapes visent à étendre ce cadre à plusieurs périodes et à mettre en œuvre les outils numériques nécessaires à une simulation complète (avec Streamlit par exemple).

\newpage

%==========================================================
% JOUR 4
%==========================================================

\section{Jour 4, fin du cours sur le modèle CRR}

\subsection{Objectifs de la journée}
\begin{itemize}
    \item Comprendre comment construire un portefeuille répliquant pour évaluer un produit dérivé.
    \item Introduire la stratégie de couverture (hedging) via la réplication.
    \item Définir la probabilité neutre au risque et comprendre son rôle.
    \item Relier la notion de \(\Delta\)-hedging à la dérivée du prix de l’option.
\end{itemize}

\subsection{Contenu abordé}

\subsubsection*{Modèle binomial à une période}
\begin{itemize}
    \item L’actif risqué vaut \(S_0\) aujourd’hui.
    \item Il peut évoluer vers \(S_0 u\) (hausse) ou \(S_0 d\) (baisse) à la prochaine période.
    \item Il existe un actif sans risque avec un taux \(r\).
\end{itemize}

\subsubsection*{Portefeuille répliquant}
\begin{itemize}
    \item On construit un portefeuille de \(\Delta\) actions et \(B\) obligations.
    \item Objectif : faire en sorte que ce portefeuille reproduise les gains du produit dérivé dans les deux cas futurs.
    \item Valeur initiale : \(V_0 = \Delta S_0 + B\).
    \item Valeur finale en cas de hausse : \(V_u = \Delta S_0 u + B(1 + r)\).
    \item Valeur finale en cas de baisse : \(V_d = \Delta S_0 d + B(1 + r)\).
\end{itemize}

\subsubsection*{Résolution du système}
On résout les deux équations :
\[
\Delta = \frac{V_u - V_d}{S_0 (u - d)}, \quad B = \frac{u V_d - d V_u}{(1 + r)(u - d)}
\]

\subsubsection*{Lien avec la couverture}
\begin{itemize}
    \item Cette stratégie est appelée \textbf{stratégie de couverture}.
    \item On annule le risque de marché en prenant une position opposée à celle de l’option.
    \item \(\Delta\) est appelé \textbf{delta} car il représente la dérivée du prix de l’option par rapport au sous-jacent.
    \item Cela fonde la notion de \textbf{delta-hedging}.
\end{itemize}

\subsubsection*{Probabilité neutre au risque}
\[
q = \frac{(1 + r) - d}{u - d}, \quad 1 - q = \frac{u - (1 + r)}{u - d}
\]

\subsubsection*{Formule de valorisation}
On obtient finalement :
\[
\boxed{V_0 = \frac{1}{1 + r} \left( q \cdot V_u + (1 - q) \cdot V_d \right)}
\]
Ce qui correspond à la valeur actualisée de l'espérance du produit dérivé sous la probabilité neutre au risque.

\subsection{À suivre...}
\begin{itemize}
    \item Développer la convergence du modèle binomial vers le modèle de Black-Scholes.
\end{itemize}

\subsection{Conclusion}
\begin{itemize}
    \item Le modèle CRR permet d’évaluer les options sans arbitrage par réplication.
    \item Le portefeuille répliquant constitue une stratégie de couverture efficace.
    \item La notion de \(\Delta\)-hedging est centrale en gestion du risque.
    \item La probabilité neutre au risque offre un cadre probabiliste puissant pour la valorisation.
\end{itemize}

\newpage

%==========================================================
% JOUR 5
%==========================================================

\section{Jour 5, Implémentation du portefeuille}
\subsection{Objectifs de la journée}
\begin{itemize}
    \item Implémenter un système de gestion de portefeuille (portefeuille multi-options)
    \item Intégrer une stratégie de couverture (delta hedging) au sein du portefeuille
    \item Ajouter des tests unitaires pour valider le bon fonctionnement des composants
\end{itemize}

\subsection{Travail réalisé}
\begin{itemize}
    \item Création du fichier \texttt{portfolio.py} pour gérer un ensemble de positions sur options.
    \item Implémentation du fichier \texttt{hedging.py} pour calculer les deltas et recommander une couverture.
    \item Résolution d’un bug lié à l’attribut \texttt{pricing\_model} manquant dans le portefeuille.
    \item Écriture de tests unitaires
\end{itemize}

\subsection{À suivre...}
\begin{itemize}
    \item Ajouter des scénarios dynamiques simulant l’évolution du marché (trajectoire du spot)
    \item Implémenter une stratégie de rebalancement du hedge à chaque pas de temps
    \item Intégrer les frais de transaction dans la stratégie de couverture
    \item Optimiser la visualisation avec des animations ou des courbes interactives
\end{itemize}

\subsection{Conclusion}
La journée a permis d’implémenter les bases d’un système cohérent de valorisation de portefeuille avec couverture delta.

\newpage

%==========================================================
% JOUR N
%==========================================================

\section{Jour N, ...}
\subsection{Objectifs de la journée}
\begin{itemize}
    \item ...
\end{itemize}
\subsection{...}
\subsection{À suivre...}
\subsection{Conclusion}
\newpage

%==========================================================
\end{document}