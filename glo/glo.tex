\documentclass[a4paper,10pt]{article}
\usepackage[utf8]{inputenc}
\usepackage[T1]{fontenc}
\usepackage[french]{babel}
\usepackage{multicol}
\usepackage[colorlinks=true, linkcolor=black]{hyperref}
\usepackage{geometry}
\geometry{margin=2cm}

\usepackage{titling}
\renewcommand\maketitlehooka{\null\mbox{}\vfill}
\renewcommand\maketitlehookd{\vfill\null}

\title{\Huge{\textbf{Modèle de Cox-Ross-Rubinstein}}\\ \medskip
      \Huge{\textit{Glossaire}}\vspace*{0.7cm}}
\author{\LARGE{Alexis VO}\vspace{1cm}\\ \medskip
      Université Paris-Saclay\\École polytechnique}
\date{\vspace{0.2cm}\today}

% === BEGIN DOCUMENT ===
\begin{document}

\vspace{\fill}
  \maketitle
\vspace{\fill}

\newpage

\tableofcontents

\newpage

\begin{multicols}{2}

\section*{A}
\addcontentsline{toc}{section}{A}
\begin{itemize}
  \item \textbf{Actif/Asset} : bien ou droit possédé par un individu, un pays ou une entreprise, qui peut générer des flux de trésorerie futurs.
  \item \textbf{Action / Share / Stock} : titre de propriété représentant une part du capital d’une entreprise. Donne des droits à son détenteur, notamment le droit de vote en assemblée générale et le droit de percevoir des dividendes.
  \item \textbf{Arbitrage} : stratégie d’investissement qui permet de réaliser un profit sans risque, en exploitant les différences de prix entre plusieurs marchés.
\end{itemize}

\section*{B}
\addcontentsline{toc}{section}{B}
\begin{itemize}
  \item \textbf{Baissier / Bearish} : terme utilisé pour décrire une tendance à la baisse des prix d’un actif financier ou du marché dans son ensemble. Un investisseur baissier s’attend à ce que les prix diminuent et peut vendre des actifs dans l’espoir de réaliser un profit.
  \item \textbf{Bénéfice / Profit} : gain réalisé par une entreprise après déduction de ses coûts et charges. Il peut être réinvesti dans l’entreprise ou distribué aux actionnaires sous forme de dividendes.
  \item \textbf{Bourse / Stock exchange} : marché organisé où s’échangent des titres financiers tels que les actions, les obligations et les produits dérivés. Exemples : NYSE, NASDAQ, Euronext.
\end{itemize}

\section*{C}
\addcontentsline{toc}{section}{C}
\begin{itemize}
  \item \textbf{Capital} : somme d’argent investie dans une entreprise ou un actif, qui peut générer des revenus.
  \item \textbf{Contrat à terme / Forward contract} : accord entre deux parties pour acheter ou vendre un actif à un prix fixé à une date future. Exemple : contrats à terme sur matières premières, devises ou indices boursiers.
  \item \textbf{Contrat d’option / Option contract} : contrat financier qui donne à son détenteur le droit, mais pas l’obligation, d’acheter ou de vendre un actif sous-jacent à un prix déterminé avant une date d’échéance. Exemples : options d’achat (call) et options de vente (put).
  \item \textbf{Cours / Price} : valeur d’un actif financier sur le marché, déterminée par l’offre et la demande. Le cours peut varier en fonction des conditions économiques, des performances de l’entreprise et des attentes des investisseurs.
  \item \textbf{Créance / Debt} : somme d’argent empruntée par une entreprise ou un individu, qui doit être remboursée avec des intérêts. Les créances peuvent prendre la forme d’obligations, de prêts bancaires ou de crédits à la consommation.
\end{itemize}

\section*{D}
\addcontentsline{toc}{section}{D}
\begin{itemize}
  \item \textbf{Delta} : mesure de la sensibilité du prix d’une option par rapport à une variation du prix de l’actif sous-jacent. Il indique le changement attendu du prix de l’option pour une variation d’une unité du prix de l’actif. Exemple : si le delta d’une option est de 0,5, une augmentation de 1 unité du prix de l’actif sous-jacent entraînera une augmentation de 0,5 unité du prix de l’option.
  \item \textbf{Derivé / Derivative} : instrument financier dont la valeur dépend de la valeur d’un actif sous-jacent, comme une action, une obligation ou une matière première.
  \item \textbf{Dividend} : part des bénéfices d’une entreprise distribuée à ses actionnaires, généralement sous forme de paiement en espèces ou d’actions supplémentaires.
\end{itemize}

\section*{E}
\addcontentsline{toc}{section}{E}
\begin{itemize}
  \item \textbf{ETF (Exchange-Traded Fund)} : fonds d’investissement coté en bourse qui suit un indice, une matière première ou un panier d’actifs. Il permet aux investisseurs d’acheter une part diversifiée d’actifs sans avoir à acheter chaque actif individuellement.
  \item \textbf{Exercise} : action de faire valoir le droit d’une option, soit en achetant l’actif sous-jacent (pour une option d’achat) ou en le vendant (pour une option de vente).
\end{itemize}

\section*{F}
\addcontentsline{toc}{section}{F}
\begin{itemize}
  \item \textbf{Finance} : domaine d’étude et de gestion des ressources monétaires dans le temps, en particulier dans des situations d’incertitude.
\end{itemize}

\section*{G}
\addcontentsline{toc}{section}{G}

\section*{H}
\addcontentsline{toc}{section}{H}
\begin{itemize}
  \item \textbf{Haussier / Bullish} : terme utilisé pour décrire une tendance à la hausse des prix d’un actif financier ou du marché dans son ensemble. Un investisseur haussier s’attend à ce que les prix augmentent et peut acheter des actifs dans l’espoir de réaliser un profit.
  \item \textbf{Hedge fund} : fonds d’investissement qui utilise des stratégies de trading avancées, telles que la vente à découvert, l’effet de levier et les produits dérivés, pour générer des rendements élevés. Les hedge funds sont généralement réservés aux investisseurs institutionnels et aux particuliers fortunés en raison de leur risque élevé.
  \item \textbf{Hedging} : stratégie de gestion du risque qui consiste à prendre une position opposée sur un actif ou un instrument financier pour compenser les pertes potentielles d’une autre position. Exemple : acheter une option de vente pour se protéger contre une baisse du prix d’une action détenue.
\end{itemize}

\section*{I}
\addcontentsline{toc}{section}{I}
\begin{itemize}
  \item \textbf{Indice / Index} : mesure statistique qui reflète la performance d’un groupe d’actifs, comme un indice boursier. Exemple : CAC 40, S\&P 500.
  \item \textbf{Instrument financier} : contrat ou titre qui représente un droit de propriété ou une créance sur un actif, comme les actions, les obligations et les produits dérivés.
  \item \textbf{Intérêt / Interest} : coût de l’emprunt d’argent, généralement exprimé en pourcentage du montant emprunté. Il peut être fixe ou variable.
  \item \textbf{Investissement / Investment} : allocation de ressources financières dans l’espoir de générer un rendement futur. Cela peut inclure l’achat d’actions, d’obligations, de biens immobiliers ou d’autres actifs.
\end{itemize}

\section*{J}
\addcontentsline{toc}{section}{J}
\begin{itemize}
  \item \textbf{Jours de bourse / Trading days} : jours pendant lesquels les marchés financiers sont ouverts pour les transactions. En général, les jours de bourse sont du lundi au vendredi, à l’exception du 1er janvier et du Vendredi Saint.
  \item \textbf{Jours ouvrés / Business days} : jours pendant lesquels les institutions financières et les entreprises sont ouvertes pour les affaires. Cela inclut généralement les jours de bourse, mais peut également inclure des jours où les marchés financiers sont fermés.
\end{itemize}

\section*{K}
\addcontentsline{toc}{section}{K}

\section*{L}
\addcontentsline{toc}{section}{L}

\section*{M}
\addcontentsline{toc}{section}{M}
\begin{itemize}
  \item \textbf{Marché / Market} : lieu ou système où les acheteurs et les vendeurs échangent des biens, des services ou des actifs financiers. Les marchés peuvent être physiques (comme une bourse) ou virtuels (comme une plateforme de trading en ligne).
  \item \textbf{Marché primaire / Primary market} : marché où les nouveaux titres financiers sont émis et vendus pour la première fois, généralement par le biais d’une introduction en bourse (IPO = Initial Public Offering).
  \item \textbf{Marché secondaire / Secondary market} : marché où les titres financiers déjà émis sont échangés entre investisseurs, sans que l’entreprise émettrice ne soit impliquée. Exemples : NYSE, NASDAQ.
  \item \textbf{Marge / Margin} : somme d’argent déposée par un investisseur pour garantir une position de trading, généralement utilisée pour emprunter des fonds supplémentaires afin d’augmenter la taille de la position.
  \item \textbf{Maturity} : date à laquelle une option ou un contrat à terme arrive à échéance et doit être exercé ou réglé.
  \item \textbf{Mécanisme de marché / Market mechanism} : ensemble des règles et des processus qui régissent les transactions sur un marché financier, y compris la formation des prix, la liquidité et la transparence.
  \item \textbf{Model} : représentation simplifiée et théorique de l’évolution d’un phénomène financier.
  \item \textbf{Modèle de Cox-Ross-Rubinstein} : modèle d’évaluation des options qui utilise un arbre binaire pour représenter les mouvements possibles du prix d’un actif sous-jacent au fil du temps. Il est basé sur l’hypothèse que les prix des actifs suivent un processus de marche aléatoire.
  \item \textbf{Modèle de Black-Scholes} : modèle d’évaluation des options qui utilise une formule mathématique pour déterminer le prix théorique d’une option européenne, en tenant compte du prix de l’actif sous-jacent, du prix d’exercice, du temps jusqu’à l’échéance, de la volatilité et du taux d’intérêt sans risque.
  \item \textbf{Monnaie / Currency} : unité de compte utilisée pour mesurer la valeur des biens et services. Exemples : dollar, euro, yen.
\end{itemize}

\section*{N}
\addcontentsline{toc}{section}{N}
\begin{itemize}
  \item \textbf{Négociation / Trading} : processus d’achat et de vente d’actifs financiers sur un marché, dans le but de réaliser un profit ou de gérer le risque.
  \item \textbf{Notation / Rating} : évaluation de la qualité de crédit d’un émetteur d’obligations ou d’autres titres, généralement effectuée par des agences de notation comme Moody’s, S\&P ou Fitch.
\end{itemize}

\section*{O}
\addcontentsline{toc}{section}{O}
\begin{itemize}
  \item \textbf{Obligation / Bond} : titre de créance émis par une entreprise ou un gouvernement pour emprunter de l’argent. L’émetteur s’engage à rembourser le montant emprunté à une date future, ainsi qu’à verser des intérêts périodiques aux détenteurs d’obligations.
  \item \textbf{Option} : contrat financier qui donne à son détenteur le droit, mais pas l’obligation, d’acheter ou de vendre un actif sous-jacent à un prix déterminé avant une date d’échéance. Les options peuvent être classées en deux catégories : les options d’achat (call) et les options de vente (put).
  \item \textbf{Option américaine / American option} : option qui peut être exercée à tout moment avant la date d’échéance, contrairement aux options européennes qui ne peuvent être exercées qu’à la date d’échéance.
  \item \textbf{Option asiatique / Asian option} : option dont le prix d’exercice est basé sur la moyenne des prix de l’actif sous-jacent sur une période donnée, plutôt que sur le prix à un moment précis.
  \item \textbf{Option européenne / European option} : option qui ne peut être exercée qu’à la date d’échéance, contrairement aux options américaines qui peuvent être exercées à tout moment avant l’échéance.
  \item \textbf{Option exotique / Exotic option} : option qui a des caractéristiques ou des conditions d’exercice plus complexes que les options standard, comme les options à barrière, les options de type lookback ou les options de type basket.
  \item \textbf{Option binaire / Binary option} : option qui ne paie qu’un montant fixe si elle est exercée, ou rien du tout si elle n’est pas exercée. Elle est souvent utilisée pour parier sur la direction du mouvement des prix d’un actif.
  \item \textbf{Option d’achat / Call option} : option qui donne le droit, mais pas l’obligation, d’acheter un actif sous-jacent à un prix déterminé avant une date d’échéance.
  \item \textbf{Option de vente / Put option} : option qui donne le droit, mais pas l’obligation, de vendre un actif sous-jacent à un prix déterminé avant une date d’échéance.
\end{itemize}

\section*{P}
\addcontentsline{toc}{section}{P}
\begin{itemize}
  \item \textbf{Paiement / Payoff} : montant reçu ou payé à l’échéance d’un contrat financier, comme une option ou un contrat à terme. Le paiement dépend des conditions du contrat et de la performance de l’actif sous-jacent.
  \item \textbf{Portefeuille / Portfolio} : ensemble d’actifs financiers détenus par un investisseur, qui peut inclure des actions, des obligations, des options et d’autres instruments financiers. La diversification du portefeuille vise à réduire le risque global en combinant différents types d’actifs.
  \item \textbf{Prix d’exercice / Strike price} : prix auquel un détenteur d’option peut acheter (pour une option d’achat) ou vendre (pour une option de vente) l’actif sous-jacent.
  \item \textbf{Prix de l’actif sous-jacent / Underlying asset price} : prix actuel de l’actif sur lequel une option est basée, comme une action, une obligation ou une matière première.
  \item \textbf{Prix de l’option / Option price} : prix payé pour acquérir une option, qui dépend de divers facteurs tels que le prix de l’actif sous-jacent, le prix d’exercice, le temps jusqu’à l’échéance et la volatilité.
\end{itemize}

\section*{Q}
\addcontentsline{toc}{section}{Q}
\begin{itemize}
  \item \textbf{Quantité / Quantity} : nombre d’unités d’un actif financier achetées ou vendues dans une transaction. Elle peut être exprimée en actions, en contrats à terme ou en lots.
  \item \textbf{Quote / Cote} : prix actuel d’un actif financier sur le marché, qui peut être affiché sous forme de prix bid (prix d’achat) et ask (prix de vente).
\end{itemize}

\section*{R}
\addcontentsline{toc}{section}{R}
\begin{itemize}
  \item \textbf{Rendement / Return} : gain ou perte réalisé sur un investissement, généralement exprimé en pourcentage du montant investi. Il peut inclure les dividendes, les intérêts et les plus-values.
  \item \textbf{Risque / Risk} : incertitude quant à la performance future d’un actif financier, qui peut entraîner des gains ou des pertes. Le risque peut être systématique (lié au marché dans son ensemble) ou spécifique (lié à un actif ou à une entreprise particulière).
  \item \textbf{Risque de crédit / Credit risk} : risque qu’un emprunteur ne rembourse pas ses dettes, entraînant des pertes pour le prêteur.
  \item \textbf{Risque de marché / Market risk} : risque que la valeur d’un actif financier diminue en raison de mouvements défavorables du marché, tels que des fluctuations de prix ou des changements de taux d’intérêt.
\end{itemize}

\section*{S}
\addcontentsline{toc}{section}{S}
\begin{itemize}
  \item \textbf{Spread} : différence entre le prix d’achat (bid) et le prix de vente (ask) d’un actif financier. Un spread plus étroit indique une liquidité plus élevée et des coûts de transaction plus faibles.
  \item \textbf{Strike price} : prix auquel un détenteur d’option peut acheter (pour une option d’achat) ou vendre (pour une option de vente) l’actif sous-jacent.
  \item \textbf{Swap} : contrat financier dans lequel deux parties échangent des flux de trésorerie futurs basés sur des actifs sous-jacents, tels que des taux d’intérêt, des devises ou des matières premières.
  \item \textbf{Système de cotation / Quotation system} : méthode utilisée pour afficher les prix des actifs financiers sur un marché, qui peut être basée sur des enchères (comme la cotation en continu) ou sur des prix fixes (comme la cotation en bloc).
  \item \textbf{Système de négociation / Trading system} : plateforme ou infrastructure utilisée pour exécuter des transactions sur un marché financier, qui peut inclure des systèmes électroniques, des courtiers ou des salles de marché.
\end{itemize}

\section*{T}
\addcontentsline{toc}{section}{T}
\begin{itemize}
  \item \textbf{Taux d’intérêt / Interest rate} : pourcentage appliqué à un montant emprunté ou investi, qui détermine le coût de l’emprunt ou le rendement de l’investissement. Il peut être fixe ou variable.
  \item \textbf{Taux de change / Exchange rate} : prix d’une monnaie par rapport à une autre, qui détermine la valeur relative des devises sur le marché des changes.
  \item \textbf{Taux de croissance / Growth rate} : pourcentage d’augmentation d’une variable économique, comme le PIB, les bénéfices d’une entreprise ou le prix d’un actif financier, sur une période donnée.
  \item \textbf{Taux de rendement / Yield} : mesure du rendement d’un investissement, généralement exprimée en pourcentage du montant investi. Il peut inclure les dividendes, les intérêts et les plus-values.
  \item \textbf{Taux de volatilité / Volatility rate} : mesure de la variation des prix d’un actif financier sur une période donnée, qui peut être utilisée pour évaluer le risque et la performance potentielle d’un investissement.
  \item \textbf{Taux sans risque / Risk-free rate} : taux de rendement d’un investissement considéré comme sans risque, généralement basé sur les obligations d’État à court terme. Il est utilisé comme référence pour évaluer le rendement des autres investissements.
\end{itemize}

\section*{U}
\addcontentsline{toc}{section}{U}
\begin{itemize}
  \item \textbf{Utilité / Utility} : mesure de la satisfaction ou du bien-être qu’un individu tire de la consommation d’un bien ou d’un service. En finance, l’utilité est souvent utilisée pour évaluer les préférences des investisseurs et leur aversion au risque.
  \item \textbf{Unité de compte / Unit of account} : unité de mesure utilisée pour exprimer la valeur des biens et services, facilitant ainsi la comparaison des prix et la comptabilité.
  \item \textbf{Unité de risque / Risk unit} : mesure utilisée pour quantifier le risque associé à un investissement ou à un portefeuille, qui peut inclure des facteurs tels que la volatilité, la corrélation et l’exposition au marché.
  \item \textbf{Unité de temps / Time unit} : période de temps utilisée pour mesurer la performance d’un investissement ou d’un actif financier, qui peut inclure des jours, des mois, des trimestres ou des années.
  \item \textbf{Unité de mesure / Unit of measure} : standard utilisé pour quantifier une variable économique ou financière, facilitant ainsi la comparaison et l’analyse des données.
  \item \textbf{Unité de risque / Risk unit} : mesure utilisée pour quantifier le risque associé à un investissement ou à un portefeuille, qui peut inclure des facteurs tels que la volatilité, la corrélation et l’exposition au marché.
\end{itemize}

\section*{V}
\addcontentsline{toc}{section}{V}
\begin{itemize}
  \item \textbf{Volatilité / Volatility} : mesure de la variation des prix d’un actif financier sur une période donnée, qui peut être utilisée pour évaluer le risque et la performance potentielle d’un investissement. La volatilité peut être historique (basée sur les prix passés) ou implicite (basée sur les prix des options).
  \item \textbf{Valeur actuelle / Present value} : valeur actuelle d’un flux de trésorerie futur, actualisée à un taux d’intérêt donné. Elle est utilisée pour évaluer la rentabilité des investissements et des projets.
  \item \textbf{Valeur liquidative / Net asset value (NAV)} : valeur totale des actifs d’un fonds d’investissement, moins ses passifs, divisée par le nombre de parts en circulation. Elle est utilisée pour évaluer la performance et la valeur d’un fonds.
\end{itemize}

\section*{W}
\addcontentsline{toc}{section}{W}
\begin{itemize}
  \item \textbf{Wall Street} : rue à New York qui abrite la Bourse de New York (NYSE) et de nombreuses institutions financières. Elle est souvent utilisée comme synonyme du secteur financier américain.
  \item \textbf{Warrant} : titre financier qui donne à son détenteur le droit d’acheter des actions d’une entreprise à un prix déterminé avant une date d’échéance. Les warrants sont souvent émis par les entreprises pour lever des fonds.
\end{itemize}

\section*{X}
\addcontentsline{toc}{section}{X}
\begin{itemize}
  \item \textbf{Xetra} : plateforme de négociation électronique de la Bourse de Francfort, qui permet aux investisseurs d’acheter et de vendre des actions, des obligations et d’autres instruments financiers.
  \item \textbf{X-Options} : options sur indices boursiers, qui permettent aux investisseurs de parier sur la performance d’un indice plutôt que sur une action individuelle.
\end{itemize}

\section*{Y}
\addcontentsline{toc}{section}{Y}
\begin{itemize}
  \item \textbf{Yield curve} : graphique représentant les taux d’intérêt des obligations à différents échéances, qui peut être utilisé pour évaluer les attentes du marché en matière de croissance économique et d’inflation.
  \item \textbf{Yield to maturity (YTM)} : taux de rendement total attendu d’une obligation si elle est conservée jusqu’à son échéance, prenant en compte les paiements d’intérêts et la différence entre le prix d’achat et la valeur nominale.
\end{itemize}

\section*{Z}
\addcontentsline{toc}{section}{Z}
\begin{itemize}
  \item \textbf{Zero-coupon bond} : obligation qui ne verse pas d’intérêts périodiques, mais est émise à un prix inférieur à sa valeur nominale. Le rendement est réalisé à l’échéance lorsque l’obligation est remboursée à sa valeur nominale.
  \item \textbf{Z-Score} : mesure statistique utilisée pour évaluer la distance d’un point de données par rapport à la moyenne d’un ensemble de données, exprimée en écarts-types. En finance, le Z-Score est souvent utilisé pour évaluer la solidité financière d’une entreprise.
  \item \textbf{Zone euro / Eurozone} : région économique composée des pays de l’Union européenne qui ont adopté l’euro comme monnaie unique. Elle est gérée par la Banque centrale européenne (BCE) et vise à promouvoir la stabilité économique et monétaire.
  \item \textbf{Zone de convergence / Convergence zone} : région géographique où les taux d’intérêt, les taux de change et les niveaux de prix convergent vers des valeurs similaires, généralement en raison de l’intégration économique et financière entre les pays.
\end{itemize}

\end{multicols}

\end{document}