\documentclass[11pt]{beamer}
\usetheme{Madrid}
\usefonttheme{serif}

\usepackage[utf8]{inputenc}
\usepackage[french]{babel}
\usepackage[T1]{fontenc}

\usepackage{amsmath, amsfonts, amssymb}
\usepackage{graphicx}
\usepackage{mathrsfs}
\usepackage{hyperref}
\usepackage{bookmark}

\usepackage{tikz}

\setbeamertemplate{caption}[numbered]
\setbeamertemplate{navigation symbols}{}

\title{Modèle de Cox-Ross-Rubinstein}
\author[Alexis VO]{Alexis VO}

\institute[]{CMAP - École polytechnique}
\date{11 juin 2025}

\begin{document}

\begin{frame}
\titlepage
\end{frame}

\begin{frame}{Plan}
\tableofcontents
\end{frame}

\section{Contexte}

\begin{frame}{Objectifs du stage}
\begin{itemize}
    \item Comprendre le modèle de Cox-Ross-Rubinstein pour la valorisation d'options.
    \item Implémenter un modèle de gestion de portefeuille en Python.
    \item Étudier la couverture dynamique (delta hedging).
    \item Vers le modèle de Black-Scholes ?
    \item ...
\end{itemize}
\end{frame}

\section{Modèle CRR}

\begin{frame}{Hypothèses}
\begin{itemize}
    \item Marché frictionless\footnote{sans coûts à l'achat/vente/détention}, sans arbitrage.
    \item Le prix suit un processus binomial :
    \[
        S_{t+\Delta t} = 
        \begin{cases}
            uS_t & \text{avec probabilité } p \\
            dS_t & \text{avec probabilité } 1 - p
        \end{cases}
    \]
    \item Probabilité neutre au risque : 
    \[
    q = \frac{(1 + r) - d}{u - d}
    \]
    \item L'espérance du rendement de l'actif sous-jacent est égale au taux sans risque :
    \[ E\left[\frac{S_{t+1}-S_t}{S_t}\right] = r \]
\end{itemize}


\end{frame}

\begin{frame}{Modèle CRR}
\begin{center}
    \begin{tikzpicture}[>=stealth, node distance=2.2cm and 3.2cm]

    \node (S0) at (0,0) {$S_0$};

    \node (Su1) at (3,2) {$S_0 \cdot u$};
    \node (Sd1) at (3,-2) {$S_0 \cdot d$};

    \node (Sun) at (7,3) {$S_0 \cdot u^n$};
    \node (Midn) at (7,0)   {$\cdots$};
    \node (Sdn) at (7,-3) {$S_0 \cdot d^n$};

    \draw[->] (S0) -- (Su1);
    \draw[->] (S0) -- (Sd1);

    \draw[dashed, ->] (Su1) -- (Sun);
    \draw[dashed, ->] (Su1) -- (Midn);
    \draw[dashed, ->] (Sd1) -- (Midn);
    \draw[dashed, ->] (Sd1) -- (Sdn);

    \node at (0,-3.5) {Temps $t = 0$};
    \node at (3,-3.5) {Temps $t = 1$};
    \node at (7,-3.5) {Temps $t = n$};

    \end{tikzpicture}
\end{center}
\end{frame}

\begin{frame}{Modèle CRR}
La valeur initiale \(V_0\) d'un produit dérivé est donné par le 
\begin{block}{Théorème d'évaluation des actifs}
    \[V_0 = \frac{1}{1 + r} \left( q \cdot V_u + (1 - q) \cdot V_d \right)\]
\end{block}
où \( q \) est la probabilité neutre au risque, \( V_u \) (resp. \(V_d\)) la valeur haussière (resp. à la baissière).

\end{frame}


\begin{frame}{Valorisation d'une option}
TODO
\begin{block}{Formule par backward induction}
\[
C_0 = e^{-r\Delta t} \left( p C_u + (1 - p) C_d \right)
\]
\end{block}

\begin{itemize}
    \item Application à une option européenne en bout d’arbre.
    \item Cas d’une option américaine : vérification de l’exercice anticipé à chaque noeud.
\end{itemize}
\end{frame}

\section{Travail réalisé}

\begin{frame}{Implémentation Python}
\begin{itemize}
    \item TODO
    \item Construction automatique de l’arbre binomial.
    \item Calcul des prix des options européennes (call/put).
    \item Visualisation interactive (matplotlib, plotly).
    \item Comparaison CRR vs Black-Scholes.
\end{itemize}
\end{frame}

\begin{frame}{Exemple de résultats}
\begin{itemize}
    \item TODO
    \item Graphe : arbre de prix
    \item Courbe : valeur de l'option selon la maturité
    \item Tableau : comparaison CRR / Black-Scholes
\end{itemize}

% Uncomment the following lines if you add the image at images/arbre_exemple.png
% \begin{center}
%     \includegraphics[width=0.6\textwidth]{images/arbre_exemple.png}
% \end{center}
\end{frame}

\section{Perspectives}

\begin{frame}{Améliorations et extensions}
\begin{itemize}
    \item Intégration des dividendes
    \item Couverture dynamique (delta hedging)
    \item Extension à un portefeuille multi-actifs
    \item Interface utilisateur (Streamlit)
\end{itemize}
\end{frame}

\section{Questions}

\begin{frame}{Questions}
\begin{center}
    \begin{itemize}
        \item Convergences et limites : existe-t-il des bornes d’erreur précises sur l’approximation du prix d’option via CRR en fonction du nombre de pas ?
        \item Choix de u et d : y a-t-il des variantes du modèle CRR qui modifient ces facteurs pour mieux refléter des comportements du marché réel ?
        \item Applications :  dans quelles situations le modèle CRR reste-t-il encore très pertinent pour la pratique financière ?
        \item Demander d'autres ressources telles que TD/TP
        \item Ressources, articles, ouvrages recommandés pour approfondir ?
    \end{itemize}
\end{center}

\vspace{0.5cm}

% \begin{figure}[htb]
%     \centering
%     \includegraphics[width=0.3\textwidth]{images/logo_cmap.png}
% \end{figure}
\end{frame}

\end{document}